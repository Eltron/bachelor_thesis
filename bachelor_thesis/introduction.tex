%%%%%%%%%%%%%%%%%%%%%%%%%%%%%%%%%%%%%%%%%%%%%%%%%%%%%%%%%%%%%%%%%%%%%%%%%%%%%%%%
\intro
Необходимость улучшить обслуживание клиентов и автоматизировать взаимодействие с ними приводит всё больше компаний к решению использовать специальные информационные системамы.

Одним из таких решений является CRM(Customer relationship management) - термин, характеризующий практики, стратегии и технологии, используемые компаниями для управления и анализа данных и взаимодействий с пользователем.\cite{crm}

Основные причины внедрения системы управления взаимоотношениями:

\begin{itemize}
	\item Ведение общей клиентской базы.
	\item Повышение эффективности продаж;
	\item Повышение эффективности маркетинга;
	\item Повышение уровня обслуживания клиентов;
	\item Отчетность и анализ деятельности отделов, связанных со взаимодействием с клиентами.
\end{itemize}

К сожалению, в большинстве таких систем возможности аналитики данных ограничиваются формированием отчётности только по вопросам отношений с клиентами (сколько привлечено клиентов, прогнозы по продажам и т.д.), чего во многих случаях не хватает для бизнес-анализа.

Для решения данной проблемы многие организации используют более мощные средства и платформы - системы бизнес-аналитики. Business Intelligence (BI) - это набор методов и инструментов для объединения, обработки и предоставления информации в удобной, осмысленной форме.

Однако, возникает проблема взаимодействия данных систем друг с другом. Наилучшим решением является интеграция CRM-системы с BI-платформой.

Целью данной работы является разработка модуля интеграции аналитических отчётов платформы Pentaho BI в CRM-систему Vtiger. Данный модуль позволит экспортировать данные из отчётов, находящихся в хранилище BI-платформы для дальнейшей работы с ними в CRM-системе.


%%%%%%%%%%%%%%%%%%%%%%%%%%%%%%%%%%%%%%%%%%%%%%%%%%%%%%%%%%%%%%%%%%%%%%%%%%%%%%%%
