%%%%%%%%%%%%%%%%%%%%%%%%%%%%%%%%%%%%%%%%%%%%%%%%%%%%%%%%%%%%%%%%%%%%%%%%%%%%%%%%
\chapter{}
\vspace{7em}
\begin{center}
	Кроссплатформенное мобильное приложение - клиент системы управления бизнес процессами Camunda.
\end{center}

\begin{center}
	Техническое задание.
\end{center}
\begin{center}
	1 лист
\end{center}

\newpage
\textbf{Введение}\\
Наименование программы: «Кроссплатформенное мобильное приложение - клиент системы управления бизнес процессами Camunda». Данное средство предназначено для управления бизнес процессами системы Camunda, через мобильное устройство. 

\textbf{Назначение разработки}\\
Обеспечение скорости и удобства сотрудников в системе Camunda BPM без привязки к рабочему месту в офисе.

\textbf{Требование к программе или программному изделию}

\textit{Требования к функциональным характеристикам}\\
Разработанное приложение должно поддерживать все возможности модуля Tasklist web-версии системы, в частности:
\begin{itemize}
	\item Создание произвольных задач или старт предварительно определенных бизнес-процессов;
	\item Получение списка текущих бизнес-процессов или задач, а также их состояния и возможностью просмотра более детальной информации;
	\item Разграничение задач по пользователям;
	\item Изменение статуса и комментирования бизнес-процессов или задач.
\end{itemize}
Производить аутентификацию в системе, с помощью логина и пароля какого-либо пользователя системы. Возможность задания адреса и порта сервера с развернутой системой Camunda BPM.

\textit{Требования к составу и параметрам технических средств}\\
Разработанное кроссплатформенное мобильное приложение может функционировать на аппаратной платформе следующих мобильных операционных системах:
\begin{itemize}
	\item Android (версия 4.0.3 и старше)
	\item IOS (версия 6.1 и старше)
\end{itemize}

\textit{Требования к информационной и программной совместимости}\\
Программа должна быть написана на языке C\# с использованием фреймворка Xamarin, и взаимодействовать с сервером Camunda через REST-интерфейс.


\chapter{}
%\addcontentsline{toc}{chapter}{ПРИЛОЖЕНИЕ Б. Описание программы}
%%%%%%%%%%%%%%%%%%%%%%%%%%%%%%%%%%%%%%%%%%%%%%%%%%%%%%%%%%%%%%%%%%%%%%%%%%%%%%%%
\vspace{7em}

\begin{center}
	Кроссплатформенное мобильное приложение - клиент системы управления бизнес процессами Camunda.
\end{center}

\begin{center}
	Описание программы.
\end{center}
\begin{center}
	1 лист
\end{center}

\newpage
\textbf{Общие сведения}\\
Разработанное кроссплатформенное мобильное приложение, реализует всю функциональность модуля Tasklist web-версии системы Camunda. Это позволяет автоматизировать работу с бизнес процессами через мобильное устройство.

\textbf{Используемые технические средства}\\
Для функционирования приложения, необходимо подключение к сети в которой находится сервер с Camunda, а также одна из следующих мобильных операционных систем:
\begin{itemize}
	\item Android (версия 4.0.3 и старше);
	\item IOS (версия 6.1 и старше).
\end{itemize} 

\textbf{Входные данные}\\
Входными данными являются сетевая конфигурация:
\begin{enumerate}
	\item адрес сервера;
	\item порт сервера.
\end{enumerate}
А также данные пользователя, необходимые для аутентификации:
\begin{enumerate}
	\item логин пользователя;
	\item пароль пользователя.
\end{enumerate}

\newpage

\chapter{}

\vspace{7em}

\begin{center}
	Кроссплатформенное мобильное приложение - клиент системы управления бизнес процессами Camunda.
\end{center}

\begin{center}
	Программа и методика испытаний.
\end{center}
\begin{center}
	3 листа
\end{center}
\newpage
\textbf{Объект испытаний}\\
Наименование программы – объекта испытаний:  «Кроссплатформенное мобильное приложение - клиент системы управления бизнес процессами Camunda».

\textbf{Цель испытаний}\\
Целью проведения испытаний является проверка соответствия функциональных характеристик средства требованиям, указанным в техническом задании.

\textbf{Требования к программе}\\
Разработанное средство должно функционировать на мобильных операционных систем:
\begin{itemize}
	\item Android (версия 4.0.3 и старше);
	\item IOS (версия 6.1 и старше).
\end{itemize} 
А также подключаться к заданному серверу с Camunda, и удалённо производить все действия модуля Tasklist с бизнес процессами.

\textbf{Средства и порядок испытаний}\\
Программа должна выполняться на эмуляторах мобильных операционных систем Android(версия 4.0.3 и старше) и IOS(версия 6.1 и старше) или на физических устройствах с соответствующей версией ОС. Эмулятор или устройство, на котором проходит тестирование, должно находится в одной сети с компьютером на котором развернута бизнес система Camunda. Для проведения испытаний был собран тестовый стенд, состоящий из трех компьютеров, находящихся в одной сети.

Компьютер с эмулятором Android:
\begin{enumerate}
	\item Оборудование:
	\begin{itemize}
		\item Intel Core i5 4210U;
		\item Оперативная память – 8 Гб;
		\item Жёсткий диск – 250 Гб;
		\item Карта сетевого интерфейса(NIC);
		\item Стандартный монитор;
		\item Клавиатура;
		\item Мышь.
	\end{itemize}
	\item Операционная система:
	\begin{itemize}
		\item Windows 10.
	\end{itemize}
	\item Установленное ПО:
	\begin{itemize}
		\item Android SDK(API 15 - API 25);
		\item Фреймворк Xamarin;
		\item Microsoft Visual Studio 2015.
	\end{itemize}
\end{enumerate}
Компьютер с эмулятором IOS:
\begin{enumerate}
	\item Оборудование:
\begin{itemize}
	\item Intel Core i5 5250U;
	\item Оперативная память – 4 Гб;
	\item Жёсткий диск – 128 Гб;
	\item Карта сетевого интерфейса(NIC);
	\item Стандартный монитор;
	\item Клавиатура;
	\item Мышь.
\end{itemize}
\item Операционная система:
\begin{itemize}
	\item Mac OS X v10.10 Yosemite.
\end{itemize}
\item Установленное ПО:
\begin{itemize}
	\item IOS SDK;
	\item Фреймворк Xamarin;
	\item Xamarin Studio.
\end{itemize}
\end{enumerate}
Компьютер с развернутой Camunda BPM:
\begin{enumerate}
	\item Оборудование:
	\begin{itemize}
		\item AMD FX 8120;
		\item Оперативная память – 6 Гб;
		\item Жёсткий диск – 500 Гб;
		\item Карта сетевого интерфейса(NIC);
		\item Стандартный монитор;
		\item Клавиатура;
		\item Мышь.
	\end{itemize}
\item Операционная система:
\begin{itemize}
	\item Linux Mint 18.1 Serena.
\end{itemize}
\item Установленное ПО:
\begin{itemize}
	\item Camunda BPM.
\end{itemize}
\end{enumerate}
\textbf{Порядок проведения испытаний:}
\begin{enumerate}
	\item запуск сервера с Camunda BPM;
	\item проверка на нахождение в одной сети всех трех компьютеров;
	\item выбор конфигурации эмуляторов Android и IOS;
	\item запуск эмуляторов Android и IOS;
	\item запуск программного средства;
	\item проверка на соответствие функциональных характеристик, указанных в техническом задании;
	\item анализ проведённых испытаний.
\end{enumerate}



\newpage

\chapter{}

\vspace{7em}

\begin{center}
	Кроссплатформенное мобильное приложение - клиент системы управления бизнес процессами Camunda.
\end{center}

\begin{center}
	Фрагменты кода программы.
\end{center}
\begin{center}
	22 листа
\end{center}
\newpage


%Листинги
%\lstinputlisting[
%label={listings:camundaProc.cs},
%caption={camundaProc.cs},
%style=Java]
%{src/camundaProc.cs}

%\lstinputlisting[
%label={listings:camundaTasks.cs},
%caption={camundaTasks.cs},
%style=Java]
%{src/camundaTasks.cs}

%\lstinputlisting[
%label={listings:camundaUser.cs},
%caption={camundaUser.cs},
%style=Java]
%{src/camundaUser.cs}



\lstinputlisting[
label={listings:createTask.xaml.cs},
caption={createTask.xaml.cs},
style=Java]
{src/createTask.xaml.cs}

%\lstinputlisting[
%label={listings:CustomWebView.cs},
%caption={CustomWebView.cs},
%style=Java]
%{src/CustomWebView.cs}

\lstinputlisting[
label={listings:engine},
caption={engine.cs},
style=Java]
{src/engine.cs}

\lstinputlisting[
label={listings:fileDownloader.cs},
caption={fileDownloader.cs},
style=Java]
{src/fileDownloader.cs}
%%%%%%%%%%%%%%%%%%%%%%%%%%%%%%%%%
\lstinputlisting[
label={listings:MainPage.xaml.cs},
caption={MainPage.xaml.cs},
style=Java]
{src/MainPage.xaml.cs}

%\lstinputlisting[
%label={listings:pdfView.xaml.cs},
%caption={pdfView.xaml.cs},
%style=Java]
%{src/pdfView.xaml.cs}

%\lstinputlisting[
%label={listings:createProc.xaml.cs},
%caption={createProc.xaml.cs},
%style=Java]
%{src/createProc.xaml.cs}

%\lstinputlisting[
%label={listings:procChoose.xaml.cs},
%caption={procChoose.xaml.cs},
%style=Java]
%{src/procChoose.xaml.cs}

%\lstinputlisting[
%label={listings:taskPage.xaml.cs},
%caption={taskPage.xaml.cs},
%style=Java]
%{src/taskPage.xaml.cs}

\lstinputlisting[
label={listings:ValueDataConverter},
caption={ValueDataConverter.cs},
style=Java]
{src/ValueDataConverter.cs}

%\lstinputlisting[
%label={listings:variable},
%caption={variable.cs},
%style=Java]
%{src/variable.cs}

%\lstinputlisting[
%label={listings:workspace.xaml.cs},
%caption={workspace.xaml.cs},
%style=Java]
%{src/workspace.xaml.cs}

%\lstinputlisting[
%label={listings:AndroidCustomWebViewRenderer.cs},
%caption={[Android]CustomWebViewRenderer.cs},
%style=Java]
%{src/android/CustomWebViewRenderer.cs}

%\*lstinputlisting[
%label={listings:AndroidworkingWithFiles.cs},
%caption={[Android]workingWithFiles.cs},
%style=Java]
%{src/android/workingWithFiles.cs}

%\lstinputlisting[
%label={listings:IOSCustomWebViewRenderer.cs},
%caption={[IOS]CustomWebViewRenderer.cs},
%style=Java]
%{src/ios/CustomWebViewRenderer.cs}

%\lstinputlisting[
%label={listings:IOSworkingWithFiles.cs},
%caption={[IOS]workingWithFiles.cs},
%style=Java]
%{src/ios/workingWithFiles.cs}

\lstinputlisting[
label={listings:UItests},
caption={UItests.cs},
style=Java]
{src/UItests.cs}

%\lstinputlisting[
%label={listings:unitTests},
%caption={UnitTests.cs},
%style=Java]
%{src/UnitTests.cs}