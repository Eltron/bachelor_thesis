\chapter{Тестирование системы}
Тестирование было разделено на \textbf{backend}(внутренняя реализация) и \textbf{frontend}(внешнее представление) части.

\section{Backend тестирование}

В тестировании класса выборки данных из массива реализован следующий алгоритм:

\begin{enumerate}
	\item Сгенерирована JSON строка;
	\item Сгенерирован JSON массив в соответствии со строкой;
	\item Произведено сравнение полученного результата с этелонной моделью, указанной вручную.
\end{enumerate} 

Для тестирования класса выборки данных использовался фреймворк JUnit.

Был создан класс TestDataSelector, часть которого представлена в листинге \ref{listings:selTest}:

\begin{lstlisting}[
label={listings:selTest},
caption={"Класс тестирования"},
style=java]
import org.junit.Test;
import static org.junit.Assert.assertEquals;

public class TestDataSelector {

	@Test
	public void selectDataTest() {
		String json = "{[[{\"value\":\"null\",\"type\":\"COLUMN_HEADER\"}, {\"value\":\"USA\",\"type\":\"COLUMN_HEADER\"},
		...
		{\"value\":\"6532\",\"type\":\"DATA_CELL\"}, {\"value\":\"4387\",\"type\":\"DATA_CELL\"}]]}";
		
		JSONObject jsonObj = new JSONObj();
		JSONArray array = new JSONArray();
		DataSelector dataSelector = new DataSelector(json);
		
		jsonObj.put(value, "Store 6");
		jsonObj.put(type, "COLUMN_HEADER");
		array.add(jsonObj);
		...
		assertEquals(dataSelector.selectData("Store 6|Drinks"), array);
	}
	...
}
\end{lstlisting}

\section{Тестирование REST-сервиса}
Jersey, используемый в проекте для тестирования предоставляет минимальный веб-сервер для развёртывания пользователем. Для тестирования разработанного сервиса необходимо проверить работу точек входа для получения массива данных и для отправки отчёта в формате JSON. 

Для закрытия сервера после исполнения всех запланированных тестов использовалась аннотация @AfterClass.

\begin{lstlisting}[
label={listings:afterClass},
caption={"Использование Аннотации AfterClass"},
style=java]
@AfterClass
public static void afterClass() throws Exception {
	server.close();
}
\end{lstlisting}

Для верификации результата теста нужно сравнить код ответа от сервера с ожидаемым(200 - OK), указанным в спецификации.

\begin{lstlisting}[
label={listings:status},
caption={"Сравнение ответного кода с ожидаемым положительным"},
style=java]
assertEquals(Response.Status.OK.getStatusCode(), response.getStatus());
\end{lstlisting}

Были протестированы варианты кодов ответа: позитивного(200), негативного(505) и случай недоступности сервиса(404).

\section{Frontend тестирование}

Тестирование интерфейса плагина, в данном случае, представляет из себя тестирование Javascript кода, отвечающего за выбор опции выпадающего списка и обработчиков нажатия на кнопку. Для тестирования использовалась библиотека Jasmine, предоставляющая возможность создавать “заглушки” для элементов:

\begin{lstlisting}[
label={listings:jasm},
caption={"Пример использования библиотеки Jasmine"},
style=java]
var exec;
beforeEach(){
	exec = jasmine.createSpy();
	spyOn(document 'getElemntById').andReturn({exec:getTable})
}
\end{lstlisting}

\section{Итоги}

Тестирование любого проекта является его важной частью. Разделение тестирования на backend и frontend части позволяет проверить работу как в серверной части, так и в части интерфейса. 

Тесты для серверной части помогают проверить корректность выборки данных из JSON массива.

Данные, полученные в тестировании, свидетельствуют о корректной работе сервиса. При некоректно выбранных данных генерация не происходит и пользователю возвращается корректная ошибка.