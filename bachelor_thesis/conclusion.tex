%%%%%%%%%%%%%%%%%%%%%%%%%%%%%%%%%%%%%%%%%%%%%%%%%%%%%%%%%%%%%%%%%%%%%%%%%%%%%%%%
\conclusion
%%%%%%%%%%%%%%%%%%%%%%%%%%%%%%%%%%%%%%%%%%%%%%%%%%%%%%%%%%%%%%%%%%%%%%%%%%%%%%%%
В процессе выполнения работы, были исследованы информационные системы, хранящие и позволяющие анализировать информацию, связанную с бизнес-процессами и менеджментом.

Исследование показало, что наибольшую эффективность представляет связка информационная система - платформа бизнес-аналитики. Такая связка даёт возможность использовать инструменты BI платформы для обработки данных из хранилища CRM.

В данной работе исследовалась возможность интеграции CRM-системы Vtiger и BI платформы Pentaho BI Suite. 

Для интеграции был написан плагин для Pentaho, дающий возможность получить необходимые данные из OLAP-отчёта. Также, разработанный плагин выводит отчёт и генерирует URL-адрес по выбранным заголовкам таблицы.

В процессе разработки возникли определённые трудности. Так, на момент разработки плагина, не удавалось получить URL для доступа к отчёту по причине нерабочей документации Saiku(ошибка 404). На поиск решения, либо замены на другой вариант, ушло слишком много времени, что впоследствии стало причиной превышения сроков выполнения работы.

Разработанный плагин действительно расширяет возможности бизнес-аналитики по сравнению со встроенными в CRM-систему инструментами. При использовании модуля появляются возможности получать результаты инструмента аналитики и прогнозирования Saiku.

